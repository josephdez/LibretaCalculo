\documentclass{book}
\usepackage[spanish]{babel}
\usepackage[utf8]{inputenc}
\usepackage{graphicx}
\usepackage{amsfonts}
\usepackage{amsmath}
\usepackage{amssymb}
\usepackage{enumerate}
\setlength\parindent{0pt}
\title{Cálculo Integral En Una Variable}
\author{José Juan Hernández Cervantes}
\date{Julio-Diciembre 2017}
\begin{document}
\maketitle
\chapter{Propiedades de los Números Reales}
\section{Axioma Del Supremo.}
Todo subconjunto no vacío de $\mathbb{R}$ acotado superiormente tiene supremo.

\textbf{Definición: Supremo.}

Sea $A \supseteq \mathbb{R} \ (A \neq \varnothing)$ un conjunto acotado superiormente.
Diremos que $\bar{x}$ es el supremo de \textit{A} (y lo denotamos por $SupA$) si cumple:
\begin{enumerate}[1.]
\item $\bar{x}$  es cota superior de $A$.
\item Si \textit{z} es cota superior de \textit{A}, ocurre $\bar{x}\le z$.
\end{enumerate}
\textbf{Teorema: Unicidad del supremo.}

Si $\bar{x}$ es el supremo de $A$, $\bar{x}$ es único.

\textbf{Demostración}

Supongamos $\bar{x}$ y $\bar{y}$ supremos de $A$. Entonces, por definición de supremo ocurre:
$\bar{x}\le\bar{y}\land\bar{y}\le\bar{x}$

$\therefore \bar{x}=\bar{y}$

\textit{Q.E.D}

\section{Propiedad Arquimedeana.}

Para todo par de números $x,y\in \mathbb{R}\mbox{ con }x> 0 \ \exists \ n\in\mathbb{N}\mbox{ tal que }nx> y$.

\textbf{Demostración: por reducción a lo absurdo.}

Supongamos $\forall \ n\in\mathbb{N},nx\le{y}$.
Si $y\le{0}$ entonces $x\le{0}$, contradicción con la hipótesis $x>0$.
Si $y>0$, sea $A=\lbrace nx:n\in{\mathbb{N}}\rbrace$.
Trivialmente $A\subseteq{\mathbb{R}}$ y $ A\neq \varnothing\mbox{ pues }x\in A$, además \textit{A} está acotado superiormente por \textit{y}.
Invocando el axioma del supremo, existe $\bar{x}=SupA$.
Como $x>0 \Rightarrow -x<0\Rightarrow \bar{x}-x<\bar{x}$.
Con lo que $\bar{x}-x$ no es cota superior de \textit{A}.
Entonces existe $a\inA$ tal que $\bar{x}-x<a$.
Esto es, $\exists \ n\in{\mathbb{N}}\mbox{ tal que }\bar{x}-x<xn=a$. Equivalentemente $\bar{x}<(n+1)x$.
Como $(n+1)x\in{A}$, llegamos a una contradicción con la definicion de supremo.

$\therefore \ \exists \ n\in{\mathbb{N}}\mbox{ tal que }nx>y$ $\forall \  x>0,y\in{\mathbb{R}}$.

\textit{Q.E.D}

\section{Principio Del Buen Orden.}

Todo subconjunto no vacío de $\mathbb{N}$ tiene elemento mínimo.

$\forall A\subseteq{\mathbb{R}}$, $ A\neq \varnothing$, $\exists \  a_0 :a_0 \le a$ $ \forall a\in{A}$.

\section{Principio De Inducción Matemática Fuerte.}

Si $A=\lbrace P(j):j\in{\mathbb{N}}\rbrace$ es una colección de enunciados con las siguientes propiedades:
\begin{enumerate}[1.]
\item \textit{P(1)} es verdadero.
\item \textit{P(n+1)} es verdadero siempre que \textit{P(n),P(n-1),...,P(2),P(1)} sean verdaderos.
\end{enumerate}
Entonces \textit{P(j)} es verdadero $\forall j\in{\mathbb{N}}$

\subsection{El principio de inducción matemática fuerte implica el principio de buen orden.}

\textbf{Demostración: por reducción a lo absurdo.}

Supongamos que existe $A\subseteq \mathbb{N}$, $A\neq \varnothing$, tal que no existe $a_0\in A$ \textit{ con } $a_0\le{a} \ \forall \ a\in{A}$.
Sea $B=\lbrace n\in{\mathbb{N}}:n\notin{A} \rbrace$. Entonces $1\notin{A}$, pues $ 1\le{n}$ $\forall \ n\in{\mathbb{N}}$.
Se sigue que $1\in{B}$ $(B\neq \varnothing)$.
Supongamos $k\in{B}$, entonces \textit{1,2,...,k-1,k} $\notin{A}$.
Luego $k+1\notin{A}$, de lo contrario \textit{k+1} sería el elemento más pequeño de \textit{A}.
Por el Principio De Inducción Matemática Fuerte tenemos $B=\mathbb{N}\mbox{, como }A\subseteq{\mathbb{N}}=B\mbox{ ocurre }A=\varnothing$.
Contradicción con la hipótesis.

$\therefore \ \forall \ A\subseteq{\mathbb{N}} \ y \ A\neq{\varnothing} \ \exists \ a_0:a_0\le{a} \ \forall \ a\in{A}$.

\textit{Q.E.D}

\subsection{El principio de buen orden implica el principio de inducción matemática fuerte.}

\textbf{Demostración: por reducción a lo absurdo.}

Supongamos $P=\lbrace P(n)/n\in{\mathbb{N}}\rbrace$ es un conjunto de propiedades tales que:
\begin{enumerate}[1.]
\item \textit{P(1)} es verdadero.
\item Siempre que para un $K\in{\mathbb{N}},P(K)$ es verdadero, entonces \textit{P(K+1)} es verdadero.
\end{enumerate}
Supongamos falso que \textit{P(n)} es verdadero $\forall$ $n\in{\mathbb{N}}$.\textit{(*)}

Entonces existe un $r\in{\mathbb{N}}$ tal que \textit{P(r)} es falso.

Sea $A=\lbrace K\in{\mathbb{N}}:P(K)\mbox{ es falso }\rbrace$ luego $A\subseteq{\mathbb{N}} \ y \ A\neq{\varnothing}$, ya que $r\in{A}$.

Por el principio del buen orden, existe $k_{0}\in{\mathbb{N}}\mbox{ tal que } \ k_{0}\le{k} \ \forall \ k\in{A}$.

Observemos que $k_{0}>1$, pues por hipótesis \textit{P(1)} es verdadero, entonces $k_{0}-1\in{\mathbb{N}} \ y \ k_{0}-1<k_{0}\mbox{ luego }k_{0}-1\not\in{A}$.

Entonces $P(k_{0}-1)$ es cierto,luego,por hipótesis, $P(k_{0})$ es verdadero.Contradicción, pues $k_{0}\in{A}$.

Llegamos a una contradicción al suponer falso \textit{(*)}.
  
\section{Teorema Del Binomio De Newton.}
$\mbox{Para cualesquiera }a,b\in{\mathbb{R}}\land \forall \ n\in{\mathbb{N}}$ se tiene:
$(a+b)^n=\sum_{k=0}^n \binom{n}{k} a^kb^{n-k}$

\textbf{Demostración: Por inducción sobre n.}

$P(0):(a+b)^0=1$.

Por otro lado, $\sum_{k=0}^0 \binom{0}{k} a^kb^{0-k}=\binom{0}{0}a^0b^0=1$.

$\therefore$ \textit{P(0)} es verdadero.

Supongamos \textit{P(n)} verdadero, es decir $(a+b)^n=\sum_{k=0}^n \binom{n}{k} a^kb^{n-k}$.
Por demostrar \textit{P(n+1)} verdadero.

$(a+b)^{n+1}=(a+b)^n \ (a+b)=(a+b) \sum_{k=0}^n \binom{n}{k} a^kb^{n-k}$

$=a\sum_{k=0}^n \binom{n}{k} a^kb^{n-k}+b\sum_{k=0}^n \binom{n}{k} a^kb^{n-k}$

$=\sum_{k=0}^n \binom{n}{k} a^{k+1}b^{n-k}+\sum_{k=0}^n \binom{n}{k} a^kb^{n-k+1}$

$=\sum_{k=1}^{n+1} \binom{n}{k-1} a^kb^{n-k+1}+\sum_{k=0}^n \binom{n}{k} a^kb^{n-k+1}$

$=\sum_{k=1}^n \binom{n}{k-1} a^kb^{k+1-r}+\sum_{k=1}^n \binom{n}{k} a^kb^{n-k+1}+\binom{n}{0}a^0b^{n+1-0}$

$=a^{n+1}+\sum_{k=1}^n(\binom{n}{k-1} a^kb^{n-k+1}+\binom{n}{k} a^kb^{n-k+1})+b^{n+1}$

$=a^{n+1}+\sum_{k=1}^n a^kb^{n-k+1}(\binom{n}{k-1}+\binom{n}{k})+b^{n+1}$

$=a^{n+1}+\sum_{k=1}^n \binom{n+1}{k}a^kb^{n-k+1}+b^{n+1}$

$=\sum_{k=0}^{n+1} \binom{n+1}{k}a^kb^{n-k+1}$

\textit{Q.E.D}

\end{document}
