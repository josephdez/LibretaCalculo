\documentclass[12pt]{book}
\usepackage[utf8]{inputenc}
\usepackage[spanish]{babel}
\usepackage{amsfonts}
\usepackage{amsmath,amssymb}
\usepackage{enumerate}

\newcommand\R{{\mathbb R}}
\newcommand\C{{\mathbb C}}
\newcommand\Q{{\mathbb Q}}
\newcommand\N{{\mathbb N}}
\newcommand\Z{{\mathbb Z}}
\providecommand{\abs}[1]{\lvert#1\rvert}

\newtheorem{teo}{Teorema}[section]
\newtheorem{lema}[teo]{Lema}
\newtheorem{defi}{Definición}[section]

\setlength\parindent{0pt}

\title{Cálculo Integral En Una Variable}
\author{José Juan Hernández Cervantes}
\date{Julio-Diciembre 2017}
\begin{document}
\maketitle
\tableofcontents
\chapter{Propiedades de los Números Reales.}
\section{Supremo.}
\begin{defi}[Supremo de un conjunto]\rm
Sea $A \subseteq \R$  no vacío acotado superiormente.
Diremos que $\bar{x}$ es el supremo de \textit{A} (y lo denotamos por $Sup(A)$) si cumple:
\begin{enumerate}[1.]
\item $\bar{x}$  es cota superior de $A$.
\item Si \textit{z} también es cota superior de \textit{A}, ocurre $\bar{x}\le z$.
\end{enumerate}
\end{defi}

\textbf{Axioma del supremo}

Todo subconjunto no vacío de números reales acotado superiormente admite supremo.

\begin{teo}[Unicidad del supremo.]\rm
Si existe $\bar{x}$ supremo de $A$, $\bar{x}$ es único.
\end{teo}

Demostración:

Por definición de supremo, si $\bar{x}$ y $\bar{y}$ son supremos de $A$, entonces ocurre:
$\bar{x}\le\bar{y}$ y $\bar{y}\le\bar{x}$

Por tricotomía se tiene $\bar{x}=\bar{y}$. $\blacksquare$

\begin{teo}[Propiedades del supremo]\rm
Si $A\subseteq{\R}$ es no vacío y acotado superiormente:


\begin{enumerate}
    \item $\bar{x} \mbox{ es supremo de } A \mbox{ si dado } \epsilon >0 \mbox{ existe } a\in A \mbox{ tal que } \bar{x}-\epsilon < a$
    \item $Sup(A \cup B)=max\{ Sup(A),Sup(B) \}$
    \item Si $a+A:=\{a+x : x \in A \}$ entonces $Sup(a+A)=a+Sup(A)$
\end{enumerate}
\end{teo}

Demostración:

Sea $\bar{x}=Sup(A) \mbox{ entonces } x \le \bar{x} \  \forall{x\in A}$, luego $a+x \le a+\bar{x}$ con lo que $a+\bar{x}$ es una cota superior de $a+A$. Se sigue $Sup(a+A) \le a+\bar{x}$, ahora, si $z$ es cota superior de $a+A$ entonces $a+x \le z \ \forall{x \in A}$ consecuentemente $x \le z-a$ entonces $z-a$ es una cota superior de $A$, asi $\bar{x} \le z-a \Rightarrow a+\bar{x} \le z \Rightarrow a+\bar{x} \le Sup(a+A)$.Probando 3. La prueba de 1 y 2 se deja como buen ejercicio para el lector. $\blacksquare$

\section{Propiedad Arquimedeana.}

Para todo par de números $x,y\in \R$ con $x> 0$,existe $n\in\N$ tal que $nx> y$.

Demostración:

Procederemos por reducción a lo absurdo. Supongamos $\forall \ n\in\mathbb{N},nx\le{y}$.

Si $y\le{0}$ entonces $x\le{0}$, contradicción con la hipótesis $x>0$.

Si $y>0$, sea $A=\lbrace nx:n\in{\mathbb{N}}\rbrace$.
Es claro que $A\subseteq{\R}$ y $ A\neq \varnothing$ pues $x\in A$, además \textit{A} está acotado superiormente por \textit{y}.
Invocando el axioma del supremo, existe $\bar{x}=SupA$.

Como $x>0 \Rightarrow -x<0\Rightarrow \bar{x}-x<\bar{x}$.
Con lo que $\bar{x}-x$ no es cota superior de \textit{A}.

Entonces existe $a \in A$ tal que $\bar{x}-x \le a$.
Es decir, existe $n\in\N$ tal que $\bar{x}-x<xn=a$. Equivalentemente $\bar{x}<(n+1)x$.
Como $(n+1)x\in{A}$, llegamos a una contradicción con la definicion de supremo. $\blacksquare$

\section{Principio Del Buen Orden.}

\begin{defi}[Elemento mínimo]\rm
Sea $A\subseteq\R$ no vacío. Decimos que $x$ es el elemento mínimo de $A$ si ocurre:
\begin{enumerate}[1.]
\item $x\in{A}$
\item $\forall \ y\in{A}$ , $x\le{y}$
\end{enumerate}
\end{defi}
\begin{teo}[Principio del buen orden]\rm
Todo subconjunto no vacío de $\N$ tiene elemento mínimo.
\end{teo}
\section{Principio De Inducción Matemática Fuerte.}

Si $A=\lbrace P(j):j\in{\mathbb{N}}\rbrace$ es una colección de enunciados con las siguientes propiedades:
\begin{enumerate}[1.]
\item \textit{P(1)} es verdadero.
\item \textit{P(n+1)} es verdadero siempre que \textit{P(n),P(n-1),...,P(2),P(1)} sean verdaderos.
\end{enumerate}
Entonces \textit{P(j)} es verdadero $\forall j\in{\mathbb{N}}$

\section{Inducción implica Buen orden.}

Demostración:

Procederemos por reducción a lo absurdo. Supongamos que existe $A\subseteq \N$, no vacío tal que no tiene elemento mínimo.
Sea $B=\lbrace n\in{\mathbb{N}}:n\notin{A} \rbrace$. Entonces $1\notin{A}$, pues $ 1\le{n}$ $\forall \ n\in{\mathbb{N}}$.
Se sigue que $1\in{B}$ $(B\neq \varnothing)$.
Supongamos $k\in{B}$, entonces \textit{1,2,...,k-1,k} $\notin{A}$.
Luego $k+1\notin{A}$, de lo contrario \textit{k+1} sería el elemento más pequeño de \textit{A}.
Por el Principio De Inducción Matemática Fuerte tenemos $B=\mathbb{N}\mbox{, como }A\subseteq{\mathbb{N}}=B\mbox{ ocurre }A=\varnothing$.
Contradicción con la hipótesis. $\blacksquare$

\section{Buen orden implica Inducción.}

Demostración:

Procederemos por reducción a lo absurdo. Supongamos $P=\lbrace P(n):n\in{\N}\rbrace$ es un conjunto de propiedades tales que:
\begin{enumerate}[1.]
\item \textit{P(1)} es verdadero.
\item Siempre que para un $K\in{\N},P(K)$ es verdadero, entonces \textit{P(K+1)} es verdadero.
\end{enumerate}
Supongamos falso que \textit{P(n)} es verdadero $\forall$ $n\in{\mathbb{N}}$.

Entonces existe un $r\in{\mathbb{N}}$ tal que \textit{P(r)} es falso.

Sea $A=\lbrace K\in{\mathbb{N}}:P(K)\mbox{ es falso }\rbrace$ luego $A\subseteq{\mathbb{N}} \ y \ A\neq{\varnothing}$, ya que $r\in{A}$.

Por el principio del buen orden, existe $k_{0}\in{\mathbb{N}}\mbox{ tal que } \ k_{0}\le{k} \ \forall \ k\in{A}$.

Observemos que $k_{0}>1$, pues por hipótesis \textit{P(1)} es verdadero, entonces $k_{0}-1\in{\mathbb{N}} \ y \ k_{0}-1<k_{0}\mbox{ luego }k_{0}-1\not\in{A}$.

Entonces $P(k_{0}-1)$ es cierto,luego,por hipótesis, $P(k_{0})$ es verdadero.

Contradicción, pues $k_{0}\in{A}$. $\blacksquare$

\section{Teorema Del Binomio De Newton.}
Para cualesquiera $a,b\in\R$ y para cualquier $\ n\in\N$ se tiene:

$$(a+b)^n=\sum_{k=0}^n \binom{n}{k} a^kb^{n-k}$$

Demostración: Por inducción sobre n.

Vereficación del caso base:

$(a+b)^0=1$.

Por otro lado, $$\sum_{k=0}^0 \binom{0}{k} a^kb^{0-k}=\binom{0}{0}a^0b^0=1$$

Con lo que \textit{P(0)} es verdadero.

Supongamos \textit{P(n)} verdadero, es decir $$(a+b)^n=\sum_{k=0}^n \binom{n}{k} a^kb^{n-k}$$
Por demostrar \textit{P(n+1)} verdadero.

$(a+b)^{n+1}=(a+b)^n \ (a+b)=(a+b) \sum_{k=0}^n \binom{n}{k} a^kb^{n-k}$

$=a\sum_{k=0}^n \binom{n}{k} a^kb^{n-k}+b\sum_{k=0}^n \binom{n}{k} a^kb^{n-k}$

$=\sum_{k=0}^n \binom{n}{k} a^{k+1}b^{n-k}+\sum_{k=0}^n \binom{n}{k} a^kb^{n-k+1}$

$=\sum_{k=1}^{n+1} \binom{n}{k-1} a^kb^{n-k+1}+\sum_{k=0}^n \binom{n}{k} a^kb^{n-k+1}$

$=\sum_{k=1}^n \binom{n}{k-1} a^kb^{k+1-r}+\sum_{k=1}^n \binom{n}{k} a^kb^{n-k+1}+\binom{n}{0}a^0b^{n+1-0}$

$=a^{n+1}+\sum_{k=1}^n(\binom{n}{k-1} a^kb^{n-k+1}+\binom{n}{k} a^kb^{n-k+1})+b^{n+1}$

$=a^{n+1}+\sum_{k=1}^n a^kb^{n-k+1}(\binom{n}{k-1}+\binom{n}{k})+b^{n+1}$
$=a^{n+1}+\sum_{k=1}^n \binom{n+1}{k}a^kb^{n-k+1}+b^{n+1}$
$=\sum_{k=0}^{n+1} \binom{n+1}{k}a^kb^{n-k+1}$ $\blacksquare$

\chapter{Sucesiones de números reales.}
\section{Límite y primeros resultados.}
\begin{defi}[Sucesión]\rm
Una sucesión de números reales es una función $f:\N \rightarrow \R$.
\end{defi}
En la practica se representa a $f$ como la lista de sus valores. Es decir, en vez de escribir $(f(1),f(2),\dots)$ escribimos $(a_{1},a_{2},\dots)$.

Es común adoptar el símbolo $(a_{j})$.

Ejemplos:
\begin{enumerate}
\item $(j)=1,2,3,\dots$
\item $(\frac{1}{j})=1,\frac{1}{2},\frac{1}{3}\dots$
\item $((-1)^j)=-1,1,-1,\dots$
\end{enumerate}
\begin{defi}[Límite de una sucesión]\rm
Decimos que $\lim\limits_{j\to\infty}(a_{j})=L$ si dado $\epsilon>0$ existe $J(\epsilon)\in\N$ tal que $\abs{a_{j}-L}<\epsilon$ $\forall{j \ge J(\epsilon)}$.

En tal caso decimos que $(a_{j})$ converje a $L$ y escribimos $(a_{j})\rightarrow L$.
\end{defi}

\begin{teo}[Unicidad del límite]\rm
Si $(a_{j})$ tiene límite, éste es único.
\end{teo}
Demostración:

Procederemos por reducción a lo absurdo. Supongamos $L_1 \mbox{ y }L_2$ limites de $(a_j)$ con $L_1 \neq L_2$. Sin perder generalidad podemos suponer $L_1<L_2$. Entonces, por definición de límite, dado $\epsilon>0$ tenemos:

$\abs{a_j-L_1}<\epsilon$ $\forall \ j\ge J_1(\epsilon)$

$\abs{a_j-L_2}<\epsilon$ $\forall \ j\ge J_2(\epsilon)$

Sean $J=max\{J_1(\epsilon),J_2(\epsilon)\}$, $h=L_2-L_1/2$, $I_1=(L_1-h,L_1+h)$ e $I_2=(L_2-h,L_2+h)$

Entonces para $j \ge J$:

$a_j \in I_1$ y $a_j \in I_2$ pero $I_1 \cap I_2 = \varnothing$. Llegamos a una contradicción al suponer $L_1 \neq L_2$, debe ser $L_1 = L_2$.
$\blacksquare$
\begin{teo}[Propiedades del límite de una sucesión]\rm\label{PropiedadesLimiteSucesion}
Sean $(a_j)$ y $(b_j)$ sucsiones de números reales con límites $L \mbox{ y } M$ respectivamentes. Entonces:

\begin{enumerate}[1)]
    \item $(\lambda a_j+b_j)$ tiene límite $\lambda L+M$ $(\lambda \in \R)$
    \item $(a_jb_j)$ tiene límite $LM$
    \item Si además $b_j \neq 0$ $\forall \ j \ge 1$ y $M \neq 0$ :
    
    $(\frac{a_j}{b_j})$ tiene límite $\frac{L}{M}$
\end{enumerate}
\end{teo}
Demostración:

Se deja como buen ejercicio para el lector. $\blacksquare$
\begin{defi}\rm
Una sucesión de números reales $(a_j)$ se dice que es acotada si existe $M>0$ tal que $\abs{a_j} \le M \ \forall{j \in \N}$.
\end{defi}
\begin{teo}\rm
Toda sucesión convergente es acotada.
\end{teo}
Demostración:

Supongamos $L= \lim\limits_{j\to\infty}(a_{j})$ y sea $\epsilon=1$. Entonces existe $J(1) \in \N$ tal que $\abs{a_j-L} \le 1 \ \forall{j \ge J(1)}$. Luego $\abs{a_j} = \abs{a_j-L+L} \le \abs{a_j-L}+\abs{L} < 1+\abs{L}$.

Sea $M=Sup(\{ \abs{a_1},\abs{a_2},\dots,\abs{a_{J(1)}},1+\abs{L} \})$. Entonces $\abs{a_j} \le M \ \forall{j \in \N}$. $\blacksquare$

\section{Convergencia.}
\begin{defi}[Sucesiones monótonas]\rm
Si $(a_j)$ es una sucesión de números reales tal que $a_j \le a_{j+1} \ \forall \ j \ge 1$ decimos que la sucesión es no decreciente (creciente).

Si en la definicion anterior se cambia la condición a $a_j \ge a_{j+1} \ \forall \ j \ge 1$, obtenemos la definición de sucesión monótona no creciente (decreciente).
\end{defi}
\begin{teo}[De convergencia monótona]\rm\label{ConvergenciaMonotona}
Sea $(a_j)$ una sucesión de números reales monótona creciente acotada superiormente y $S=\{a_1,a_2,a_3, \dots \}$.

Entonces $(a_j) \rightarrow Sup(S)$.
\end{teo}
Demostración:

Sea $\bar{s}=Sup(S)$. Entonces, por definición de supremo, $a_j \le \bar{s} \ \forall \ j$. Luego para $\epsilon >0$ , $\bar{x}-\epsilon$ no es cota superior de $S$, con lo que existe $J(\epsilon) \in \N$ tal que $\bar{x}-\epsilon \le a_{J(\epsilon)}$. Asi pues, para $j \ge J(\epsilon)$ se tiene $a_j \in (\bar{s}-\epsilon,\bar{s}+\epsilon)$, es decir $\abs{a_j-\bar{s}} \le \epsilon$. $\blacksquare$

\begin{teo}["Squeezing Theorem"]\rm\label{squeezing}
Sean $(a_j) \mbox{ tal que } a_j \ge 0 \ \forall \ j$ y $(b_j) \mbox{ tal que } a_j \le b_j \ \forall j$ con $(b_j) \rightarrow 0$.
Entonces $(a_j) \rightarrow 0$.
\end{teo}
Demostración:

Como $0 \le a_j \le b_j$, ocurre $\abs{a_j-0} \le \abs{b_j-0} \mbox{ y como } (b_j) \rightarrow 0$, existe $J(\epsilon) \in \N$ tal que $\abs{b_j-0}<\epsilon \ \forall \ j \ge J(\epsilon)$, con lo que $\abs{a_j-0}<\epsilon$. $\blacksquare$
\begin{teo}["Sandwich Theorem"]\rm
Sean $(a_j),(b_j),(c_j)$ sucesiones de números reales tales que $b_j \le a_j \le c_j \ \forall j\ge 1$, con $(b_j),(c_j) \rightarrow L$. Entonces $(a_j) \rightarrow L$.
\end{teo}
Demostración:

Como $b_j \le a_j \le c_j$, entonces $0 \le a_j-b_j \le c_j-b_j$.

Sean $(d_j)=(a_j-b_j) \ge 0$ y $(e_j)=(c_j-b_j) \ge d_j$.

Luego por el teorema (\ref{PropiedadesLimiteSucesion}) parte 1, $(e_j) \rightarrow 0$ (*). Y por el teorema (\ref{squeezing}) $(d_j) \rightarrow 0$ (**).

Por (*) existe $J_1(\epsilon) \in \N \mbox{ tal que } \abs{d_j-0}<\frac{\epsilon}{2}$

Por (**) existe $J_2(\epsilon) \in \N \mbox{ tal que } \abs{b_j-L}<\frac{\epsilon}{2}$

Finalmente, si $j \ge max\{J_1(\epsilon),J_2(\epsilon)\}$, se tiene  $\abs{a_j-L}=\abs{a_j-b_j+b_j-L} \le \abs{dj-0}+\abs{b_j-L} \le \frac{\epsilon}{2}+\frac{\epsilon}{2}=\epsilon$. $\blacksquare$

\begin{teo}\rm
Sea $(a_j)$ una sucesión de números reales tal que $a_j=f(b_j)$ con $f$ una función continua y $(b_j) \rightarrow L$. Entonces $(a_j) \rightarrow f(L)$.
\end{teo}
Demostración:

Como $f \mbox{ continua, dado }\epsilon>0 \mbox{ existe } \delta(\epsilon)>0 \mbox{ tal que } \abs{f(a)-f(b)}<\epsilon$ para $\abs{a-b}<\delta(\epsilon)$.

Como $(b_j) \rightarrow L$, $\abs{b_j-L}<\delta(\epsilon) \  \forall j \ge J(\epsilon)$

Combinando lo anterior se obtiene $\abs{f(b_j)-f(L)}=\abs{a_j-f(L)}<\epsilon \ \forall j \ge J(\epsilon)$.$\blacksquare$

\section{Teorema de Bolzano-Weierstrass.}
\begin{defi}[Subsucesión]\rm
Sea $(a_j)$ una sucesión de números reales y $j_1 \le j_2 \le \dots j_k \le \dots$ una sucesión de números naturales estrictamente creciente. Entonces la sucesión $(\Tilde{a_{j_k}})$ es llamada una subsucesión de $(a_j)$.
\end{defi}
\begin{teo}\rm
Si $(a_j) \rightarrow L$, cualquier subsucesión $(\Tilde{a_{j_k}}) \rightarrow L$
\end{teo}
Demostración:

Como la sucesión converge, dodo $\epsilon >0 \mbox{ existe } J(\epsilon) \in \N$ tal que si $j \ge J(\epsilon)$ entonces $\abs{a_j-L} < \epsilon$.

Como $j_1 \le j_2 \le \dots j_k \le \dots$ es una sucesión creciente de números naturales, se verifica sin dificultad por inducción que $j_k \ge k$, con lo que si $j \ge J(\epsilon)$ entonces $j_k \ge k \ge J(\epsilon) \Rightarrow \abs{a_{j_k}-L}<\epsilon$. $\blacksquare$
\begin{defi}[Pico]\rm
Dada una sucesión $(a_j)$ de números reales, decimos que $a_k$ es un pico de la sucesión si $a_k \ge a_j \ \forall{j \ge k}$.
\end{defi}
\begin{lema}\rm
Toda sucesión $(a_j)$ de números reales admite una subsucesión monótona.
\end{lema}
Demostración:

Supongamos que $(a_j)$ tiene una cantidad infinita de picos. Entonces la subsucesión correspondiente a los picos es una sucesión monótona decreciente.

Supongamos ahora que $(a_j)$ tiene una cantidad finita de picos. Sea $K$ el último pico y $k_1=K+1$. Luego $k_1$ no es un pico, lo que implica la existencia de un $k_2>k_1 \mbox{ con } a_{k_2}>a_{k_1}$. Nuevamente $k_2>K$ no es pico lo que implica la existencia de un $k_3>k_2 \mbox{ con } a_{k_3}>a_{k_1}$.Repitir este proceso conduce a una subsucesión infinita monótona creciente. $\blacksquare$
\begin{teo}[Bolzano-Weierstrass]\rm\label{BolzanoWe}
Toda sucesión de numeros reales acotada admite una subsucesión convergente.
\end{teo}
Demostración:

Por el lema anterior, dada una sucesión de números reales acotada, ésta admite una subsucesión monótona igualmente acotada y por el teorema (\ref{ConvergenciaMonotona}), esta subsucesión converge. $\blacksquare$

\section{Criterio de Cauchy.}
\begin{defi}[Sucesión de Cauchy]\rm
Sea $(a_j)$ una sucesión de números reales. Decimos que la sucesión es de Cauchy si para todo $\epsilon>0$ existe $J(\epsilon) \in \N \mbox{ tal que } \abs{a_j-a_k}< \epsilon \ \forall \ j,k \ge J(\epsilon)$. 
\end{defi}

\begin{lema}\rm
Toda sucesión de Cauchy es acotada.
\end{lema}
Demostración:

Sea $(a_j)$ una sucesión de Cauchy y $\epsilon=1$. Entonces si $j \ge J(1)$ entonces $\abs{a_j-a_{J(1)}}<1$. Luego para $j \ge J(1)$ se tiene $\abs{a_j} \le \abs{a_{J(1)}}+1$.

Sea $M=Sup(\{ \abs{a_1},\abs{a_2}, \dots , \abs{a_{J(1)}}+1 \})$. Se sigue $\abs{a_j} \le M \ \forall{j \in \N}$. $\blacksquare$
\begin{teo}[Criterio de convergencia de Cauchy]\rm
Una sucesión de números reales $(a_j)$ es convergente si y sólo si es de Cauchy.
\end{teo}
Demostración:

Veamos que es una condición necesaria. Es decir, supongamos $(a_j) \rightarrow L$ y verifiquemos $(a_j)$ de Cauchy.

Como $(a_j) \rightarrow L$, para $\epsilon >0 \mbox{ existe } J(\epsilon) \in N \mbox{ tal que } \abs{a_j-L} < \frac{\epsilon}{2}$. Con lo que si $j,k \ge J(\epsilon)$ se tiene :

$\abs{a_j-a_k}=\abs{(a_j-L)+(L-a_k)} \le \abs{a_j-L}+\abs{a_k-L} < \frac{\epsilon}{2}+\frac{\epsilon}{2}=\epsilon$.

Veamos ahora que es una condición suficiente. Es decir, supongamos $(a_j)$ de Cauchy y verifiquemos $(a_j)$ convergente.

Como $(a_j)$ de Cauchy, por el lema anterior $(a_j)$ es acotada, y por el teorema \ref{BolzanoWe} existe $(\Tilde{a_{j_k}})$ una subsucesión convergente, digamos $(\Tilde{a_{j_k}}) \rightarrow L$. Mostraremos $(a_j) \rightarrow L$.

Como $(a_j)$ de Cauchy, dado $\epsilon>0 \mbox{ existe } J=J(\frac{\epsilon}{2}) \in \N \mbox{ tal que si } j,k \ge J$ entonces $\abs{a_j-a_k}<\frac{\epsilon}{2}$.

Luego, como $(\Tilde{a_{j_k}}) \rightarrow L$, existe $K \ge J$ perteneciente al conjunto $\{ a_1,a_2,\dots \}$ tal que $\abs{a_k-L}<\frac{\epsilon}{2}$.

Finalmente, por la desigualdad triangular $$\abs{a_j-L}=\abs{a_j-a_k+a_k-L} \le \abs{a_j-a_k}+\abs{a_k-L} <\frac{\epsilon}{2} + \frac{\epsilon}{2} =\epsilon $$
$\blacksquare$

\clearpage
La idea de una sucesión de números reales se puede extender hacia sucesiones de funciones. En el siguiente capítulo se considerarán únicamente funciones continuas de variable real.

Al conjunto de funciones continuas de $[a,b] \rightarrow \R$ se le denota por $C([a,b],\R)$.
\chapter{Sucesiones de funciones continuas.}
\begin{defi}\rm
Una sucesione de funciones continuas es una aplicación $S:\N \rightarrow C([a,b],\R)$. En la práctica se denota por $(f_n)$.
\end{defi}
\section{Convergencia puntual.}
\begin{defi}\rm
Dado $x \in I=[a,b]$, se dice que la sucesión de funciones $(f_n)$ converge puntualmente en $x$ si la sucesión de números reales $(f_n(x))$ es convergente.

Al conjunto $C$ de todos los puntos $x \in I$ en los que la sucesión de funciones $(f_n)$ converge puntualmente se le llama \textbf{campo de convergencia puntual}. Simbólicamente: $$C=\{ x \in I : (f_n(x))\mbox{  }converge \}$$

Bajo el supuesto $C \neq \varnothing$, la función $f:C \rightarrow R$ definida por $$f(x)= \lim\limits_{n\to\infty}(f_n)$$ es llamada \textbf{función límite puntual} de la sucesión $(f_n)$.
\end{defi}

Para entender la definición de convergencia puntual es importante diferenciar entre la sucesión de funciones $(f_n)$ con la sucesión de números reales $(f_n(x))$. Recuerde que en una sucesión la variable siempre es $n \in \N$ y nunca es $x \in I$. Así la sucesión $(f_n(x))$ es la aplicación que a cada número natural $n$ le asocia el número real $f_n(x)$ donde \textbf{x está fijo}.
\section{Convergencia uniforme.}
\begin{defi}\rm
Sea $J$ un intervalo no vacío contenido en el campo de convergencia puntual de la sucesión $(f_n)$ y sea $f$ su función límite puntual. Se dice que la sucesión de funciones $(f_n)$ converge uniformemente a $f$ en $J$ si dado $\epsilon>0 \mbox{ existe } N(\epsilon) \in \N$ tal que $\abs{f_n(x)-f(x)}<\epsilon \ \forall{n \ge N(\epsilon)}$. 
\end{defi}
\end{document}
