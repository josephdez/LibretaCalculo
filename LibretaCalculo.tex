\documentclass{book}
\usepackage[spanish]{babel}
\usepackage[utf8]{inputenc}
\usepackage{graphicx}
\usepackage{amsfonts}
\usepackage{amsmath}
\usepackage{enumerate} 
\title{Cálculo Integral En Una Variable}
\author{José Juan Hernández Cervantes}
\date{Julio-Diciembre 2017}
\begin{document}
\maketitle
\chapter{Propiedades de los Números Reales}
\section{Axioma Del Supremo}

Todo subconjunto no vacío de R acotado superiormente tiene supremo.

Definición : Supremo

Sea $\mbox{A} \supseteq \mbox{\mathbb{R}}$ y $(\mbox{A} \neq \varnothing )
$ acotado superiormente.

Diremos que $\bar{x}$  es el supremo de $A$ si cumple:

1.- $\bar{x}$  es cota superior de $A$


2.- Si $z$ es cota superior de $A$, ocurre $\bar{x}\le z$


Teorema : Unicidad del supremo.

Si $\bar{x}$ es el supremo de $A$, $\bar{x}$ es único.

Demostración Supongamos $\bar{x}$ y $\bar{y}$ supremos de
$A$. Entonces, por definición de supremo ocurre:


$\bar{x}\le\bar{y}\land\bar{y}\le\bar{x}$


$\Rightarrow \bar{x}=\bar{y}

Q.E.D$


\section{Propiedad Arquimedeana.}

Para todo par de números $x,y\in \mathbb{R}$ con $x> 0$ $\exists n\in\mathbb{N}$ tal que $nx> y$

\end{document}
